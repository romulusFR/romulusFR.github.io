\documentclass[12pt,a4paper]{article}
\usepackage[french]{babel} 
\usepackage[utf8]{inputenc}
\usepackage[T1]{fontenc}
\usepackage{xspace}
\usepackage{palatino}
\usepackage[]{hyperref} 
\hypersetup{
  allbordercolors=0 0 1,
  colorlinks=true,
  urlcolor=blue
}

\usepackage{enumitem}
\setlist[itemize]{label={\textbullet}}
\setlist{nolistsep}

\usepackage[paper=a4paper, hmargin=1cm, vmargin=1cm]{geometry}
\usepackage{fancyhdr}
\pagestyle{empty}

\usepackage{xcolor}

\definecolor{gris10}{rgb}{0.75,0.75,0.75}
\definecolor{gris60}{rgb}{0.60,0.60,0.60}
\definecolor{gris80}{rgb}{0.40,0.40,0.40}
\definecolor{MSBlue}{rgb}{.204,.353,.541}
\definecolor{MSLightBlue}{rgb}{.31,.506,.741}
\definecolor{LinkRed}{rgb}{.686,.231,.49}

\newcommand{\hrs}{\textcolor{gris80}{\rule{24pt}{0.5pt}}}
\newcommand{\hr}{\textcolor{gris80}{\rule{\linewidth}{0.5pt}}}

\hypersetup{
      pdfauthor   = {Romuald THION},%
      pdftitle    = {Page personnelle},%
      % pdfsubject  = {Compétences, emplois et formations},%
      pdfkeywords = {informatique, science, recherche scientifique, enseignement, maître de conférences, bases de données, sécurité, big data, modélisation, université}%
}

\usepackage{titlesec}
\titleformat{\section}{\centering\huge\bfseries\color{MSBlue}}{\thesection}{1em}{}
\titlespacing{\section}{0pt}{0pt}{0em}
\titleformat{\subsection}{\large\bfseries\color{MSLightBlue}\MakeUppercase}{\thesubsection}{1em}{}
\titlespacing{\subsection}{0pt}{16pt}{0em}
\titlespacing{\subsubsection}{0pt}{6pt}{0em}

\newcommand{\activite}[1]{\textbf{#1}\xspace}
\newcommand{\comment}[1]{\textsl{#1}\xspace}


\newcommand{\INSAL}{\textsc{Insa de Lyon}\xspace}
\newcommand{\INRIARA}{\textsc{Inria Grenoble -- Rh\^one-Alpes}\xspace}
\newcommand{\UCBL}{\textsc{Universit{\'e} Claude Bernard Lyon 1}\xspace}
\newcommand{\UNC}{\textsc{Universit{\'e} de la Nouvelle-Calédonie}\xspace}
\newcommand{\LIRIS}{\textsc{Liris}\xspace}
\newcommand{\ATER}{\textsc{ater}\xspace} 
\newcommand{\TIW}{\textsc{m2tiw}\xspace}
\newcommand{\MIF}{\textsc{master}\xspace}
\newcommand{\LIC}{\textsc{licence}\xspace}
\newcommand{\DRAAF}{\textsc{Draaf AuRA}\xspace}


\title{{Romuald THION}}
% \author{Cool Dude}
% \date{\today}

\begin{document}
% \hr

% \vspace{0.5em}

% \begin{minipage}[c]{0.5\textwidth}
  % \begin{center}
    % {\LARGE\textsc{Romuald THION}}\\
    Maître de conférences en informatique
  % \end{center}
% \end{minipage}
% \begin{minipage}[c]{0.5\textwidth}
  % \begin{center}
    % \href{mailto:romuald.thion@unc.nc}{\nolinkurl{romuald.thion@unc.nc}}\\
    % \href{mailto:romuald.thion@agriculture.gouv.fr}{\nolinkurl{romuald.thion@agriculture.gouv.fr}}\\
    
    \url{https://romulusfr.github.io/}\\ % / \url{https://github.com/romulusFR/}.
    \href{mailto:romuald.thion@agriculture.gouv.fr}{romuald.thion@agriculture.gouv.fr}
    % \href{mailto:romuald.thion@univ-lyon1.fr}{\nolinkurl{romuald.thion@univ-lyon1.fr}}\\
    % \href{https://github.com/romulusFR/}{\nolinkurl{https://github.com/romulusFR/}}\\
    
    \DRAAF, 165 rue Garibaldi, Lyon\\
    Bureau E70, téléphone 06 58 83 04 82
    % \UNC, Nouméa\\
    % Bâtiment S, bureau S25, tél. +687 290.277
    % \UCBL, Bâtiment Nautibus, Villeurbanne\\
    % bureau 237, poste (+33 4.72.4)314.34
    %\href{https://liris.cnrs.fr/romuald.thion/}{\nolinkurl{https://liris.cnrs.fr/romuald.thion/}}
  % \end{center}
% \end{minipage}

% \vspace{0.5em}

\hr

\begin{center}
  % \emph{Maître de conférences en informatique depuis 2010, je suis spécialiste en gestion des données et en sécurité. Je suis actuellement en délégation à l'Université de la Nouvelle Calédonie (UNC).}
  \emph{
    Je suis Maître de conférences en informatique depuis 2010 à l'\UCBL~(UCBL), je suis spécialiste en gestion des données et en sécurité, j'ai évolué vers les statistiques et la géomatique.
  J'ai été en délégation à l'\UNC~(UNC) en 2021 et 2022.
  Je suis désormais détaché auprès du Service Statistique Ministériel (SSM) de l'agriculture et de l'alimentation (\href{https://ssm-agriculture.github.io/}{page GitHub}) depuis septembre 2024 comme référent national des traitements géomatiques mutualisés.
  Ci-après, des ressources réalisés pendant ces activités : matériel pédagogique, exposés, travaux de recherche.
  }
\end{center}


% \hr
% \subsection*{Activités SSM de l'agriculture et de l'alimentation}

% \activite{Référent National des Traitements Géomatiques Mutualisés}\\
% \comment{\url{https://ssm-agriculture.github.io/}, \url{https://agreste.agriculture.gouv.fr}, membre.}

\hr
\subsection*{Ressources UCBL 2022-2024}

\activite{Café développeur·se LIRIS : SQL moderne}\\
\comment{\url{https://romulusfr.github.io/Modern-SQL/Modern-SQL}, auteur.}

\activite{INF3054L Conception Et Programmation Web (L3)}\\
\comment{\url{http://lifweb.pages.univ-lyon1.fr/}, auteur.}

\activite{INF1116M Cryptographie et Sécurité (M1)}\\
\comment{\url{http://mif29-crypto-sec.pages.univ-lyon1.fr/}, co-auteur.}

\activite{INF2476M Fiabilité et Sécurité des Applications (TIW-FSA) (M2 TIW)}\\
\comment{\url{http://tiw-fsa.pages.univ-lyon1.fr/}, co-auteur.}

% \activite{INF2030L Programmation Fonctionnelle (L2, LIF-PF)}\\
% \comment{\url{https://forge.univ-lyon1.fr/programmation-fonctionnelle/lifpf/}, intervenant.}

% \activite{INF2028L Bases de Données et programmation Web (L2 printemps, LIF-BDW)}\\
% \comment{\url{https://perso.univ-lyon1.fr/nicolas.lumineau/ue/bdw/}, intervenant.}

\hr
\subsection*{Ressources UNC 2021-2022}

\activite{Évolution du niveau d'équipement des ménages NC et l'influence de la mine}\\
\comment{\url{https://romulusfr.github.io/isee-rp19/}, étude statistique sur les bases ménage, co-auteur.}

\activite{Matrice de desserte en Nouvelle-Calédonie}\\
\comment{\url{https://github.com/romulusFR/desserte_NC}, calcul d'isochrones pour applications économétriques, auteur.}

\activite{Calculs et visualisation de cartes cognitives}\\
\comment{\url{https://github.com/romulusFR/cnrt_cartes_cog}, \emph{dataviz} pour de mots librement énoncés, auteur.}

\activite{Exposé sur la science informatique}\\
\comment{\url{https://github.com/romulusFR/expose-science-informatique}, à destination des lycéens, auteur.}

\activite{27\_0187 Bases de données avancées 2 (BDAV-2) (L3)}\\
\comment{\url{https://romulusfr.github.io/unc-s6-bdav-2/}, auteur.}

\activite{M1-MIAGE Bases de Données AVancées (BDAV) (M1-Miage)}\\
\comment{\url{https://romulusfr.github.io/unc-miage-bdav/}, auteur.}

\activite{27\_0171 Introduction au Web et Interface Homme/Machine (WebIHM (L1))}\\
\comment{\url{https://romulusfr.github.io/unc-s2-web-ihm/}, auteur.}

\activite{27\_0198 Développement Web (DevWeb) (L2)}\\
\comment{\url{https://romulusfr.github.io/unc-s4-devweb/}, auteur.}

\hr
\subsection*{Ressources UCBL 2010-2021}

\activite{Timeline Informatique (jeu de carte)}\\
\comment{\url{https://github.com/romulusFR/timeline_informatique} (\href{https://perso.liris.cnrs.fr/romuald.thion/files/Communication/Timeline/nine_cards_by_page.pdf}{rendu final PDF}), tout public, auteur.}

\activite{DIU-EIL Lyon 1, Programmation avancée et bases de données}\\
\comment{\url{https://forge.univ-lyon1.fr/diu-eil/bloc4}, à destination des enseignants de lycée, auteur.}

\activite{Bases de données avancées (L3)}\\
\comment{\url{https://www.youtube.com/channel/UCp6q0_DxUdNXhoc2Oue1w9g}, co-auteur.}

\hr
\subsection*{Publications scientifiques}

Listes sur \href{https://cv.hal.science/romuald-thion}{HAL} ou \href{https://dblp.org/pid/89/6675.html}{DBLP}.

\hr
\subsection*{Curriculum Vitae détaillé}

Voir le \href{./CV_2024.pdf}{document PDF}.

\end{document}
